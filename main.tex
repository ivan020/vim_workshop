\documentclass{beamer}

% Use a minimal theme
\usetheme{Madrid}

% Dark mode customization
\usecolortheme{dove}
\setbeamercolor{background canvas}{bg=black} % Black background
\setbeamercolor{normal text}{fg=white} % White text
\setbeamercolor{title}{fg=cyan} % Cyan title color
\setbeamercolor{block title}{bg=gray, fg=white} % Block titles
\setbeamercolor{block body}{bg=black, fg=white} % Block content
\setbeamercolor{item}{fg=cyan} % Bullets in cyan
\setbeamercolor{frametitle}{fg=cyan}
\setbeamercolor{section in toc}{fg=cyan}
\setbeamercolor{subsection in toc}{fg=cyan}
% Title slide info
\title{NeoVIM workshop}
\author{Ivan Evdokimov}
\date{\today}

\begin{document}

% Title slide
\begin{frame}
    \titlepage
\end{frame}

% Outline
\begin{frame}{Outline}
    \begingroup
    \setbeamercolor{section number projected}{fg=cyan} % If section numbers exist
    \setbeamercolor{section in toc}{fg=cyan}
    \setbeamercolor{subsection in toc}{fg=cyan}
    \tableofcontents
    \endgroup
\end{frame}

% Section 1
\section{Introduction}
\begin{frame}{WhatIsNeoVIM}
    \frametitle{But what is "Neo" and "VI" and "M"?}
    \begin{itemize}
        \item "Vi" is the original editor.
        \item "M" is the modification (improved).
        \item "Neo" is yet another modification of the "VIM".
    \end{itemize}
\end{frame}

\begin{frame}{WhyUseNeoVIM}
    \frametitle{Why bother with Neovim?}
    \begin{itemize}
        \item Very-very light and therefore fast.
        \item Async loading of plugins makes it blazingly fast.
        \item Having both hands on keyboard makes you even faster.
        \item You really are in control of everything in your IDE.
    \end{itemize}
\end{frame}

\begin{frame}{VimChallenges}
    \frametitle{Why \textbf{not} to use NeoVIM?}
    \begin{itemize}
        \item It takes time and effort to build the muscle memory.
        \item Requires tinkering and setting.
        \item VIM-motions are supported in other editors (i.e. VScode, nano, etc.)
    \end{itemize}
\end{frame}

\begin{frame}{WhichNeoVIM}
    \frametitle{Which VIM?}
    \begin{itemize}
        \item We will use NeoVIM.
        \item The most recent iteration in the vim-family.
        \item Community-driven.
        \item \textsf{Lua} -- the programming language to support plugins for Neovim.
    \end{itemize}
\end{frame}

% Blocks
\section{Modes}
\begin{frame}{CommandMode}
    \frametitle{Modes}
    Vim has 4 key modes:
    \begin{itemize}
        \item NORMAL
        \item COMMAND
        \item EDITING
        \item VISUAL
    \end{itemize}
\end{frame}

\begin{frame}{NORMAL}
    \frametitle{NORMAL}
    NORMAL is the default vim mode.
    \begin{itemize}
        \item Can always use ESC or CTRL+C keys to return to NORMAL.
        \item Commonly used to: redo (u), motions (h,j,k,l).
        \item Also used for file navigation:
            \begin{itemize}
                \item \textsf{M} move cursor to middle.
                \item \textsf{L} move cursor to the bottom of the display.
                \item \textsf{H} move cursot to the head of the display.
                \item \textsf{zz} to center a screen.
            \end{itemize}
    \end{itemize}
\end{frame}

\begin{frame}{COMMAND}
    \frametitle{COMMAND}
    COMMAND is used to run commands.
    You type `:' to start a command, with most used:
    \begin{itemize}
        \item \textsf{:w} to write a file.
        \item \textsf{:w filename.txt} write a file under `filename.txt' filename.
        \item \textsf{:q} to exit a file.
        \item \textsf{:q!} to exit without save.
        \item \textsf{:wq} to save and exit.
        \item \textsf{:e filename2.txt} to open `filename2.txt' from withing current editor.
        \item \textsf{:v} to split current editor.
        \item \textsf{:Ex} to open the current tree buffer.
    \end{itemize}
\end{frame}

\section{Motions}
\begin{frame}{Motions}
    \frametitle{Motions}
    \begin{itemize}
        \item Horizontal: \textsf{h} = \leftarrow, \textsf{l} = \rightarrow.
        \item Vertical: \textsf{j} = \uparrow, \textsf{k} = \downarrow.
        \item You can combine them to jump several times, try: \textsf{2k},\textsf{2l}, etc.
        \item TRY: do the challenge `basic\_motions.txt'.
    \end{itemize}
\end{frame}

\begin{frame}{AdvancedHMotions}
    \frametitle{Advanced Horizontal Motions}
    \begin{itemize}
        \item \textsf{w} jump to the \textbf{beginning} of the next token.
        \item \textsf{e} jump to the \textbf{end} of the next token.
        \item \textsf{b} jump to the \textbf{previous} token.
        \item \textsf{\_} jump to the beginning of the line.
        \item \textsf{\$} jump to the end of the line.
        \item You can also combine those with numbers, try: \textsf{2b}, \textsf{5e}.
        \item TRY: do the challenge `advanced\_motions.txt'.
    \end{itemize}
\end{frame}

\begin{frame}{MoreAdvancedHMotions}
    \frametitle{More Advanced Horizontal Motions}
    \begin{itemize}
        \item \textsf{f} + `;' to jump to the next `;' char (can be replaced with any other char).
        \item \textsf{t} + `;' to jump the char before the `;'.
        \item in these modes, you can type `;' to jump to the next occurace of this character.
        \item \textsf{F} and \textsf{T} for the backward search.
    \end{itemize}
\end{frame}

\end{frame}
\begin{frame}{AdvancedVMotions}
    \frametitle{Advanced Vertical Motions}
    \begin{itemize}
        \item \textsf{Ctrl} + \textsf{u} jump 35 lines up.
        \item \textsf{Ctrl} + \textsf{d} jump 35 lines down.
        \item \textsf{\{} to go up to the next blank line.
        \item \textsf{\}} to go down to the next blank line.
        \item TIP: use \textsf{zz} to center screen around cursor.
    \end{itemize}
\end{frame}

\section{Editing}
\begin{frame}{EnterEditing}
    \frametitle{Enter the Editing Mode}
    \begin{itemize}
        \item \textsf{i} to \textsf{i}nsert before cursor.
        \item \textsf{a} to insert \textsf{a}fter cursor.
        \item \textsf{I} to \textsf{I}nsert at the beginnig of the line.
        \item \textsf{A} to insert \textsf{a}t the end of the line.
        \item \textsf{o} to start editing from the new line.
        \item \textsf{O} to start editing from the previous line.
        \item \textsf{Ctrl} + \textsf{C} or \textsf{Esc} to exit to normal mode.
    \end{itemize}
\end{frame}

\section{Visual}
\begin{frame}{VisualMode}
    \frametitle{Visual Mode}
    \begin{itemize}
        \item commonly used to edit a document.
        \item \textsf{v} in normal mode to enter the visual mode.
        \item you can use a number of shortcuts to select a token.
        \item \textsf{viw} to highlight a word.
        \item \textsf{V} to highlight the whole line.
    \end{itemize}
\end{frame}

\section{Tips}
\begin{frame}{VimTips}
    \frametitle{Useful Tips}
    \begin{itemize}
        \item there are some very useful vim-combinations that we can use.
        \item \textsf{diw} delete a token, stay in normal mode after.
        \item \textsf{ciw} delete a token and enter the editing mode.
        \item \textsf{daW} delets a token and its surrounding to the next token.
        \item \textsf{caW} deletes a token and its surrounding to the next token, entering the edit mode.
        \item \textsf{y} will copy under the cursor.
        \item \textsf{p} will past what you copied.
        \item \textsf{/} to find a token in the document.
        \item TIP: use `.' to repeat the previous command.
        \item TRY: use `./challenges/editing\_patters.txt' to checkout these commands.
    \end{itemize}
\end{frame}

\begin{frame}{VimTips}
    \frametitle{Copy-Paste}
    \begin{itemize}
        \item vim does not use global clipboard for copy-paste operations.
        \item \textsf{y} and \textsf{p} will only work within a document.
        \item if you want to paste from the buffer, you need \textsf{Ctrl} + \textsf{Shift} + \textsf{v} combination.
    \end{itemize}
\end{frame}

\begin{frame}{VimTips}
    \frametitle{Search and Replace}
    \begin{itemize}
        \item you can search and replace using a combination in the normal mode.
        \item \textsf{:\%s/!/?/g} will replace all `!' with `?', globally.
        \item \textsf{:\%s/!/?/gc} will replace and ask you whether you want to replace.
        \item TRY: the `./searchAndReplace.txt' for exercise.
    \end{itemize}
\end{frame}

\begin{frame}{VimTips}
    \frametitle{Search and Replace with Regex}
    \begin{itemize}
        \item Regex allows you to make a search more complex.
        \item You can lookup duplicates, digits, upper/lower/CamelCase, etc.
        \item Let us try to remove a some duplicates in `./challenges/remove\_duplicates.txt'
    \end{itemize}
\end{frame}

\section{Customization}
\begin{frame}{Custom}
    \frametitle{Customization}
    \begin{itemize}
        \item Neovim is the canvas, you're the artist.
        \item You can build your own setup.
        \item You can download other people's setup (github).
        \item You can build the pre-installed configurations.
    \end{itemize}
\end{frame}
% Conclusion
\section{Conclusion}
\begin{frame}{Conclusion}
    \begin{itemize}
        \item Vim is giga-cool.
    \end{itemize}
\end{frame}

% Thank You Slide
\begin{frame}
    \centering
    {\Huge \textcolor{cyan}{Thank You!}}
\end{frame}

\end{document}

