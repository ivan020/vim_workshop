\documentclass{beamer}

% Use a minimal theme
\usetheme{Madrid}

% Dark mode customization
\usecolortheme{dove}
\setbeamercolor{background canvas}{bg=black} % Black background
\setbeamercolor{normal text}{fg=white} % White text
\setbeamercolor{title}{fg=cyan} % Cyan title color
\setbeamercolor{tableofcontents}{fg=cyan}
\setbeamercolor{block title}{bg=gray, fg=white} % Block titles
\setbeamercolor{block body}{bg=black, fg=white} % Block content
\setbeamercolor{item}{fg=cyan} % Bullets in cyan
\setbeamercolor{frametitle}{fg=cyan}

% Title slide info
\title{NeoVIM workshop}
\author{Ivan Evdokimov}
\date{\today}

\begin{document}

% Title slide
\begin{frame}
    \titlepage
\end{frame}

% Outline
\begin{frame}{Outline}
    \tableofcontents
\end{frame}

% Section 1
\section{Introduction}
\begin{frame}{WhatIsNeoVIM}
    \frametitle{But what is "Neo" and "VI" and "M"?}
    \begin{itemize}
        \item "Vi" is the original editor.
        \item "M" is the modification (improved).
        \item "Neo" is yet another modification of the "VIM".
    \end{itemize}
\end{frame}

\begin{frame}{WhyUseNeoVIM}
    \frametitle{Why bother with Neovim?}
    \begin{itemize}
        \item Very-very light and therefore fast.
        \item Async loading of plugins makes it blazingly fast.
        \item Having both hands on keyboard makes you even faster.
        \item You really are in control of everything in your IDE.
    \end{itemize}
\end{frame}

\begin{frame}{VimChallenges}
    \frametitle{Why \textbf{not} to use NeoVIM?}
    \begin{itemize}
        \item It takes time and effort to build the muscle memory.
        \item Requires tinkering and setting.
        \item VIM-motions are supported in other editors (i.e. VScode, nano, etc.)
    \end{itemize}
\end{frame}

\begin{frame}{WhichNeoVIM}
    \frametitle{Which VIM?}
    \begin{itemize}
        \item We will use NeoVIM.
        \item The most recent iteration in the vim-family.
        \item Community-driven.
        \item \textsf{Lua} -- the programming language to support plugins for Neovim.
    \end{itemize}
\end{frame}

% Blocks
\section{Command}
\begin{frame}{CommandMode}
    \frametitle{Command Mode}
    Vim has 4 key modes:
    \begin{itemize}
        \item NORMAL
        \item COMMAND
        \item EDITING
        \item VISUAL
    \end{itemize}
\end{frame}

\begin{frame}{NORMAL}
    \frametitle{NORMAL}
    NORMAL is the default vim mode.
    \begin{itemize}
        \item Can always use ESC or CTRL+C keys to return to NORMAL.
        \item You normally use shortcuts in NORMAL mode. 
        \item Commonly used: redo (u), motions (h,j,k,l).
        \item 
    \end{itemize}
\end{frame}

\section{Motions}
\begin{frame}{Motions}
    \frametitle{Motions}
    \begin{itemize}
        \item Horizontal: \textsf{h} = \leftarrow, \textsf{l} = \rightarrow.
        \item Vertical: \textsf{j} = \uparrow, \textsf{k} = \downarrow.
        \item You can combine them to jump several times, try: \textsf{2k},\textsf{2l}, etc.
        \item TRY: do the challenge `basic\_motions.txt'.
    \end{itemize}
\end{frame}

\begin{frame}{AdvancedHMotions}
    \frametitle{Advanced Horizontal Motions}
    \begin{itemize}
        \item \textsf{w} jump to the \textbf{beginning} of the next token.
        \item \textsf{e} jump to the \textbf{end} of the next token.
        \item \textsf{b} jump to the \textbf{previous} token.
        \item \textsf{\_} jump to the beginning of the line.
        \item \textsf{\$} jump to the end of the line.
        \item You can also combine those with numbers, try: \textsf{2b}, \textsf{5e}.
        \item TRY: do the challenge `advanced\_motions.txt'.
    \end{itemize}
\end{frame}

\end{frame}
\begin{frame}{AdvancedVMotions}
    \frametitle{Advanced Vertical Motions}
    \begin{itemize}
        \item \textsf{Ctrl} + \textsf{u} jump 35 lines up.
        \item \textsf{Ctrl} + \textsf{d} jump 35 lines down.
        \item \textsf{\{} to go up to the next blank line.
        \item \textsf{\}} to go down to the next blank line.
        \item TIP: use \textsf{zz} to center screen around cursor.
    \end{itemize}
\end{frame}

\section{Editing}
\begin{frame}{EnterEditing}
    \frametitle{Enter the Editing Mode}
    \begin{itemize}
        \item \textsf{i} to \textsf{i}nsert before cursor.
        \item \textsf{a} to insert \textsf{a}fter cursor.
        \item \textsf{I} to \textsf{I}nsert at the beginnig of the line.
        \item \textsf{A} to insert \textsf{a}t the end of the line.
    \end{itemize}
\end{frame}

% Conclusion
\section{Conclusion}
\begin{frame}{Conclusion}
    \begin{itemize}
        \item Beamer themes are highly customizable.
        \item Dark mode improves eye comfort in low-light environments.
        \item Use contrast carefully to keep slides readable.
    \end{itemize}
\end{frame}

% Thank You Slide
\begin{frame}
    \centering
    {\Huge \textcolor{cyan}{Thank You!}}
\end{frame}

\end{document}

